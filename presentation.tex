\documentclass{beamer}

\usepackage[spanish]{babel}
%\usepackage[utf8]{inputenc}
\usepackage{pgfgantt}

\usepackage{tikz}
\usetikzlibrary{shapes.geometric, arrows.meta, fit, backgrounds, positioning, matrix, decorations.pathreplacing, calc}
\tikzstyle{startstop} = [rectangle, rounded corners, minimum width=3cm, minimum height=1cm,text centered, draw=black, fill=red!50]
\tikzstyle{io} = [trapezium, trapezium left angle=70, trapezium right angle=110, minimum width=3cm, minimum height=1cm, text centered, draw=black, fill=blue!30]
\tikzstyle{process} = [rectangle, minimum width=3cm, minimum height=1cm, text centered, draw=black, fill=orange!30]
\tikzstyle{decision} = [diamond, minimum width=3cm, minimum height=1cm, text centered, draw=black, fill=green!20]
\tikzstyle{database} = [cylinder, minimum width=3cm, minimum height=2cm, text centered, shape border rotate=90, aspect=0.25, draw=black, fill=yellow!30]
\tikzstyle{arrow} = [thick, ->, >=stealth]
\tikzstyle{line} = [-Latex]

\newcommand{\inline}[2]{%
    \begin{tikzpicture}[baseline=(word.base), txt/.style={shape=rectangle, inner sep=0pt}]
        \node[txt] (word) {\texttt{#1}};
        \node[above] at (word.north) {\footnotesize{#2}};
    \end{tikzpicture}%
}

\usetheme{metropolis}

\title{Estudio sobre Sistemas de Recomendación y Predicción basados en el procesamiento del lenguaje natural}
\date{\today}
\author{Hugo Ferrando Seage}
\institute{Universidad Europea de madrid\\Escuela de Arquitectura, Ingeniería y Diseño}

\begin{document}
  \maketitle

  \section{Introducción}
  \begin{frame}{Introducción}
      Los recomendadores son una parte esencial de cualquier servicio de Video on Demand (VOD). Tanto Netflix como Movistar+, Amazon Hulu y HBO cuentan con sus propios sistemas.

      También existen webs que usan sus recomendadores como IMDb o FilmAffinity. Incluso existen servicios comerciales que se dedican a productivizar su sistema de recomendación, como Jinni.
  \end{frame}

  \begin{frame}{Introducción}
      Existen tres grandes tipos de sistemas de recomendación:
      \begin{itemize}
          \item Filtrado Colaborativo
          \item Filtrado por contenido
          \item Sistemas híbridos
      \end{itemize}
  \end{frame}

  \begin{frame}{Filtrado Colaborativo}
      Consiste en emparejar usuarios que tengan gustos similares y recomendar en base a esos datos.

      Normalmente se representa usando una matriz bidimensional donde las filas representan usuarios y la columnas representan productos.

      Los usuarios deben puntuar los contenidos, o se pueden usar otras metricas.
  \end{frame}

  \begin{frame}{Filtrado por Contenido}
      Consiste en la creación de un modelo que determina la similitud entre productos en base a algún criterio.

      Ese criterio puede ser cualquier elemento del producto. Para películas puede ser el género. Para restaurantes el tipo de cocina. Etc.
  \end{frame}

  \begin{frame}{Filtrado Híbrido}
      Usan una combinación de ambas técnicas para complementar las recomendaciones.
  \end{frame}

  \section{Objetivos}
  \begin{frame}{Objetivos}
      \begin{itemize}
          \item Contruir un recomendador de películas
          \item Crear el modelo en base a tres algoritmos
              \begin{itemize}
                  \item LSA
                  \item Doc2Vec
                  \item E-Modelo
              \end{itemize}
          \item Comparar y optimizar modelos
          \item Crear una interfáz desde donde poder probarlos
      \end{itemize}
  \end{frame}

  \section{Metodología}
  \begin{frame}{Metodología}
      La metodología usada ha sido ágil, basada en MVPs.

      \begin{figure}[!htbp]
          \resizebox{0.45\textwidth}{!}{
              \centering
              \begin{ganttchart}[
                  hgrid,
                  vgrid,
                  time slot format=isodate-yearmonth,
                  compress calendar
                  ]{2016-9}{2017-7}
                  \setganttlinklabel{f-s}{}

                  \gantttitlecalendar{year, month} \\
                  \ganttbar{Investigación}{2016-09}{2016-11} \\
                  \ganttbar{Doc2Vec}{2017-03}{2017-03} \\
                  \ganttbar{LSA MVP1}{2016-10}{2016-11} \\
                  \ganttbar{LSA MVP2}{2016-12}{2017-01} \\
                  \ganttbar{LSA MVP3}{2017-02}{2017-03} \\
                  \ganttbar{ALS MVP1}{2016-09}{2016-12} \\
                  \ganttbar{ALS MVP2}{2017-01}{2017-02} \\
                  \ganttbar{ALS MVP3}{2017-03}{2017-03} \\
                  \ganttbar{Desarrollo Interfaz}{2017-04}{2017-05} \\
                  \ganttmilestone{Fin Desarrollo}{2017-05} \\
                  \ganttbar{Documentación}{2017-05}{2017-06} \\
                  \ganttmilestone{Entrega \& Presentación}{2017-06}{2017-06}

                  \ganttlink{elem0}{elem1}
                  \ganttlink{elem0}{elem2}

                  \ganttlink[link type=f-s]{elem2}{elem3}
                  \ganttlink[link type=f-s]{elem3}{elem4}

                  \ganttlink[link type=f-s]{elem5}{elem6}
                  \ganttlink[link type=f-s]{elem6}{elem7}

                  \ganttlink[link type=f-s]{elem1}{elem8}
                  \ganttlink[link type=f-s]{elem4}{elem8}
                  \ganttlink[link type=f-s]{elem7}{elem8}

                  \ganttlink{elem8}{elem9}
                  \ganttlink{elem10}{elem11}
              \end{ganttchart}
          }
      \end{figure}

  \end{frame}

  \section{Descarga de datos}
  \begin{frame}{Descarga de datos}
      Para entrenar los modelos es necesario obtener una gran cantidad de textos. Para ello se ha creado un crawler usando Spidy, que descarga información y críticas de las top 1000 películas de IMDb.
  \end{frame}
  \begin{frame}{Descarga de datos}
      \begin{figure}[!htbp]
          \resizebox{0.45\textwidth}{!}{
              \begin{tikzpicture}[node distance=2cm]
                  \centering
                  \node (init) [startstop] {Top 1000 IMDb};
                  \node (link1) [decision, below of=init] {URL};
                  \node (pelicula-init) [startstop, below of=link1] {Película};
                  \node (link2) [decision, below of=pelicula-init, yshift=-2cm] {URL};
                  \node (keywords) [process, right of=link2, xshift=2cm] {Palabras clave};
                  \node (info) [process, above of=keywords] {Información};
                  \node (reviews) [startstop, below of=link2] {Críticas};
                  \node (link3) [decision, below of=reviews] {URL};
                  \node (reviews-1) [process, right of=reviews, xshift=2cm] {Crítica};
                  \node (database) [database, right of=link1, xshift=3cm] {BD};
                  \begin{scope}[on background layer]
                      \node (bbox) [rectangle,draw,minimum width=2cm] [fit = (info) (keywords) (reviews-1),fill=yellow!30,label=above:Película] {};
                  \end{scope}

                  \draw [line] (init) -- (link1);
                  \draw [line] (link1) to [bend left] node[anchor=east] {página} (init);
                  \draw [line] (link1) -- node[anchor=north] {¿visitado?} (database);
                  \draw [line] (database) |- node[anchor=west] {no} (pelicula-init);
                  \draw [line] (pelicula-init) -- (link2);
                  \draw [line] (pelicula-init) |- (info);
                  \draw [line] (link2) -- (keywords);
                  \draw [line] (link2) -- (reviews);
                  \draw [line] (reviews) -- (link3);
                  \draw [line] (link3) to [bend left] node[anchor=east] {página} (reviews);
                  \draw [line] (link3) -| (reviews-1);
              \end{tikzpicture}
          }
      \end{figure}
  \end{frame}

  \section{Limpieza de textos}
  \begin{frame}{Limpieza de textos}
      Antes de crear los modelos es necesario hacer un pretratado de los textos.
  \end{frame}
  \begin{frame}[fragile]{Ejemplo}
      \begin{verbatim}
          Zeus is a Greek God.
      \end{verbatim}
  \end{frame}
  \begin{frame}[fragile]{POS Tagger}
      \begin{figure}[!htbp]
          \centering
          \inline{Zeus}{NNP} \inline{is}{VBZ} \inline{a}{DT} \inline{Greek}{NN} \inline{God}{NNP}.
      \end{figure}
  \end{frame}
  \begin{frame}[fragile]{Hiperónimos}
      \begin{verbatim}
          Zeus is a country deity.
      \end{verbatim}
  \end{frame}
  \begin{frame}[fragile]{Nombres propios}
      Es necesario eliminar los nombres propios para que no se relacionen películas con personajes que tienen el mismo nombre.
      \begin{verbatim}
          is a country deity.
      \end{verbatim}
  \end{frame}
  \begin{frame}[fragile]{Stopwords}
      \begin{verbatim}
          country deity
      \end{verbatim}
  \end{frame}
  \begin{frame}[fragile]{Stemmer}
      \begin{verbatim}
          counti deiti
      \end{verbatim}
  \end{frame}

  \section{LSA}
  \begin{frame}{LSA}
      Latent Semantic Analysis trata de extraer conceptos de cada texto y analizar la relación entre documentos.
  \end{frame}
  \begin{frame}[fragile]{TF-IDF}
      Term Frequency-Inverse Document Frequency calcula lo relevante que es cada palabra del vocabulario dentro de cada texto.

      \begin{figure}[!htbp]
          \centering
          \[tfidf =
              \begin{tikzpicture}[baseline=-0.65ex,scale=0.8]
                  \matrix [matrix of math nodes,left delimiter=(,right delimiter=),row sep=0.5cm,column sep=0.5cm] (m) {
                      0.39&0.16&0.19&0.01&0.25&0.79&0.27 \\
                      0.12&0.12&0.06&0.46&0.21&0.07&0.83 \\
                      0.46&0.55&0.15&0.55&0.22&0.27&0.11 \\
                      0.00&0.60&0.51&0.00&0.00&0.60&0.00 \\
                      0.41&0.00&0.35&0.83&0.00&0.00&0.00 \\
                  };
                  \node[above=10pt of m-1-1, rotate=45, yshift=3mm, xshift=3mm] (top-1) {says};
                  \node[above=10pt of m-1-2, rotate=45, yshift=3mm, xshift=3mm] (top-2) {just};
                  \node[above=10pt of m-1-3, rotate=45, yshift=3mm, xshift=3mm] (top-3) {room};
                  \node[above=10pt of m-1-4, rotate=45, yshift=3mm, xshift=3mm] (top-4) {dead};
                  \node[above=10pt of m-1-5, rotate=45, yshift=3mm, xshift=3mm] (top-5) {asks};
                  \node[above=10pt of m-1-6, rotate=45, yshift=3mm, xshift=3mm] (top-6) {ship};
                  \node[above=10pt of m-1-7, rotate=45, yshift=3mm, xshift=3mm] (top-7) {mother};

                  \node[left=12pt of m-1-1] (left-1) {The Matrix};
                  \node[left=12pt of m-2-1] (left-2) {Alien};
                  \node[left=12pt of m-3-1] (left-3) {Serenity};
                  \node[left=12pt of m-4-1] (left-4) {Casablanca};
                  \node[left=12pt of m-5-1] (left-5) {Amelie};
              \end{tikzpicture}
          \]
      \end{figure}
  \end{frame}
\end{document}
